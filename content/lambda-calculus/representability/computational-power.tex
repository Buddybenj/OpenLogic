% Part: lambda-calculus
% Chapter: representability
% Section: lambda-computable

\documentclass[../../../include/open-logic-section]{subfiles}

\begin{document}

\olfileid{lc}{rep}{cp}
\olsection{Computational power}

Not only partial recursive functions are representable by lambda
terms, the inverse is true too. That is, lambda-representable
functions are partial recursive.

\begin{thm}
\ollabel{thm:lambda-computable}
If a partial function $f$ is represented by a lambda term, it is
partial recursive.
\end{thm}
\begin{proof}
  Suppose a function~$f$ is represented by a lambda term~$M$. A 
  rigid proof would go like this:
  
  First we invent a method of encoding lambda terms into natural
  numbers.

  Then we define a partial recursive function $normalize(t)$ operating on
  a lambda term (encoded as a number as above) accepted as argument, trying to
  normalize it, returning the normal form if it has, undefined otherwise.

  Then define two partial recursive functions $toChurch$ and $fromChurch$ that's able to translate a
  natural number to and from the corresponding church numeral (the
  church numeral is again encoded into a natural number as above).

  Finally we define the partial recursive function $f$ from $M$: it accepts
  arguments $\vec x$, which are translated to church numerals
  $toChurch(\vec x)$, appended to
  $M$ forming a new term $M \vec{\num{x}}$; call
  $normalize$ on this new term and wait for it to be normalized;
  finally call $fromChurch$ on the normalized term and return the
  result.

  We assume the reader is convinced of the computational power of
  partial recursive functions and that all the mentioned functions
  above are indeed representable by partial recursive functions. We
  omitted details here because including them would be
  too tedious and of little educational value.
\end{proof}

This theorem, along with the theorem we didn't prove in the last
section, establishes lambda calculus as a
computational model equally powerful as partial recursive functions
and other models like Turing Machine, of each of which the set of
definable functions is known as ``computable functions''.
\end{document}
