% Part: lambda-calculus
% Chapter: representability
% Section: general-recursive-functions

\documentclass[../../../include/open-logic-section]{subfiles}

\begin{document}

\olfileid{lc}{rep}{gen}
\olsection{General recursive functions}

Although we haven't stated it, but analogous to Kleene's Normal Form
Theorem for partial recursive functions, general recursive functions
can also be written in normal forms as follows:
\begin{align*}
  f(x) \simeq U(\mu s \; T(e,x,s))
\end{align*}
except that $T$ should be a regular function such that $\mu s \; T(e,x,s)$
is total. Thus, to prove all general recursive functions
representable in lambda calculus, it suffices to prove:

\begin{enumerate}
\item All primitive recursive functions are representable.
\item Composition of a primitive recursive function after a
  unbounded search on a regular primitive recursive function
  is representable, given the representation of both.
\end{enumerate}

The first was proved in the last section, we will prove the second in
this section.

Given the representation of a regular primitive recursive
$T(e,x,s)$ and a primitive recursive $U(x)$, the representation of
$\mu s \; T(e,x,s)$ can be:
\begin{align*}
  \num{\mu x \; f(x,z)} &= (Y ~ search) \num{f} \num{0} \num{z} \\
  search &= \lambd[g][\lambd[fxz][iszero (\num{f} x z) x (g (succ ~ x)
           z)]]
\end{align*}
where $Y$ is any fixpoint combinator. Informally speaking, $search$ is
a self-referencing function: starting with $\num{x}$, test whether $\num{f} x
z$ is zero: if so return $x$ instantly, otherwise call itself
with $succ ~ x$. Thus $(Y ~ search) \num{f} \num{0} \num{z}$ returns
the least $x$ for which $f(x,z)$ is zero.

We now prove the correctness of this representation:
\begin{lem}
  The above representation indeed represents the unbound search for
  a regular primitive recursive function, that is,
  If $f(x,z) = 0$ for some $x$ and $f(x',z) > 0$ for all $x' < x$,
    then $(Y ~ search) \num{f} \num{0} \num{z} \red \num{x}$.
\end{lem}
\begin{proof}
  We observe that $(Y ~ search) \num{f} \num{x} \num{z}$ reduces to $\num{x}$ if
  $f(x,z)=0$, or reduces to $(Y ~ search) \num{f} \num{x+1} \num{z}$
  otherwise. The lemma follows immediately.
\end{proof}

The composition of a primitive recursive function $U(x)$ and an
unbounded search on regular primitive recursive function is simply
defined as:
\begin{align*}
  \num{U(\mu s \; T(e,x,s))} &= \lambd[x][\num{U} \num{\mu s \; T(e,x,s)}]
\end{align*}

The proof of the correctness is left to the reader.

\begin{prob}
  Justify the above representation; you don't have to worry if the
  search is undefined as $T$ is regular.
\end{prob}

\end{document}