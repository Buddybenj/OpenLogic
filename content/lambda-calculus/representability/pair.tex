% Part: lambda-calculus
% Chapter: representability
% Section: pair

\documentclass[../../../include/open-logic-section]{subfiles}

\begin{document}

\olfileid{lc}{rep}{pai}
\olsection{Pair}

The pair of $M$ and $N$ (written $\tuple{M,N}$) is defined as follows:
\begin{align*}
  \tuple{M,N} &\eqs \lambd[f][fMN]
\end{align*}
Intuitively it is a function that accepts a function, passes to it the
two inner terms and returns the result of it. Following this idea we
have this constructor, receiving two terms and returns the pair
containing them:
\begin{align*}
  {pair} &\eqs \lambd[mn][\lambd[f][fmn]]
\end{align*}

And its counterpart: two access functions, receiving a pair and
returning the first or second elements in it:
\begin{align*}
  {fst} &\eqs \lambd[p][p(\lambd[mn][m])]\\
  {snd} &\eqs \lambd[p][p(\lambd[mn][n])]
\end{align*}

\begin{prob}
  Explain how the two access functions works.
\end{prob}

Now with pairs we can write out the predecessor:
\begin{align*}
  {pred} &\eqs \lambd[n][{fst}(n (\lambd[p][\tuple{{snd}~ {p}, {succ}({snd} ~{p})}]) \tuple{0, 0})]
\end{align*}
Remember that $n f x$ is defined to be $f^{n}(x)$; in this
case $f$ is a function that accepts a pair $p$ and returns a new
pair containing the second component of $p$ and successor of the
second component; $x$ is the pair $\tuple{0,0}$. Thus, the
result is $\tuple{0,0}$ for $n=0$, and $\tuple{{n-1}, n}$
otherwise. And we only return the first component of the result.

And now the subtraction, defined as $pred$ applied on $a$ for $b$ times:
\begin{align*}
  {sub} &\eqs \lambd[ab][b {pred} ~ a]
\end{align*}
\end{document}
