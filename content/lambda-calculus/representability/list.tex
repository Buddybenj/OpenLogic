% Part: lambda-calculus
% Chapter: representability
% Section: list

\documentclass[../../../include/open-logic-section]{subfiles}

\begin{document}

\olfileid{lc}{rep}{lst}
\olsection{List}

The list $[M_1, M_2, \ldots, M_n]$ is defined to be:
\begin{align*}
  [M_1, M_2, \ldots, M_n] &= \lambd[sz][s M_1 (s M_2 (\ldots (s M_n z)))]
\end{align*}

Informally speaking it's the accumulator of the list, in the sense
that it's a function accepting two terms; the first is a function that receives accumulated value
along with a new element and returns the new accumulated value; the
second is the initial accumulated value; it accumulate elements one by
one and finally returns the result.

We can easily define some useful functions based on this encoding; 
\begin{align*}
  {sum} &= \lambd[l][l ~ (\lambd[xa][add ~ x ~ a])~  0]\\
  len &= \lambd[l][l ~ (\lambd[xa][add ~ a ~ 1]) ~ 0]
\end{align*}

$sum$ calculates the sum of a list of church numerals. It works by
doing an accumulation on the list, where the initial value is $0$ and
for each element, with $x$ to be the element and $a$ to be the
accumulated value, return $add ~ x ~ a$ as the new value. The result
is the sum. Also you can $\eta$-reduce this term (we keep this form for
easier understanding) into $\lambd[l][l ~ add ~ 0]$.

\begin{prob}
  What does $len$ do? Explain.
\end{prob}

\begin{prob}
  Define a function that reverses a list.
\end{prob}
\end{document}

