% Part: lambda-calculus
% Chapter: representability
% Section: partial-recursive-functions

\documentclass[../../../include/open-logic-section]{subfiles}

\begin{document}

\olfileid{lc}{rep}{par}
\olsection{Partial recursive functions}

In this section we will discuss the representation of partial
recursive functions. It differs from general recursive function in
that the functions in unbounded search are not necessarily
regular, removing the total-ness and bringing us troubles. For example, we may want to
reuse the representation of composition of total
functions; for simplicity let's only consider the case of one inner
function of one argument:
\begin{align*}
  \num{f(g(x))} &\eqs \lambd[x][\num{f} (\num{g} x)]
\end{align*}
however this is problematic: consider when $g(x)$ is
undefined, meaning $\num{g} x$ has no normal
form (assuming representation of $g$ works); In most cases this makes
$\num{f} (\num{g} x)$ has no normal forms either, which is what we want; but consider when
$\num{f}$ is $\lambd[x][\lambd[y][y]]$, in which case $\num{f}
(\num{g} x)$ has normal form of $\lambd[y][y]$, not
satisfying our definition of representability.

This problem has been studied and some workarounds has be invented, 
which are, however, somewhat more complicated and less intuitive than the
approach we have taken. As this is a introductory text, we only hope to
give the reader a taste of how lambda calculus represents things, thus
these workarounds are omitted here. Interested reader can find other
reference texts for the actual working method.

Still, we present the theorem here without proof:
\begin{thm}
  All partial recursive functions are lambda representable.
\end{thm}

\end{document}

