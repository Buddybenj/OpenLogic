% Part: lambda-calculus
% Chapter: church-rosser
% Section: parallel-beta-reduction

\documentclass[../../../include/open-logic-section]{subfiles}

\begin{document}

\olfileid{lc}{cr}{pb}

\olsection{Parallel $\beta$-reduction}

In this section we introduce \emph{parallel $\beta$-reduction}, and
prove it satisfying Church-Rosser property.

\begin{defn}[parallel $\beta$-reduction, $\bredpar$] \ollabel{def}
  Parallel reduction ($\bredpar$) of terms is inductively defined as follows:
  \begin{enumerate}
    \item \ollabel{def:1} $x \bredpar x$.
    \item \ollabel{def:2} If $N \xrightarrow{\beta} N'$ then $\lambd[x][N] \bredpar
      \lambd[x][N']$.
    \item \ollabel{def:3} If $P \bredpar P'$ and $Q \bredpar Q'$ then $PQ \bredpar
      P'Q'$.
    \item \ollabel{def:4} If $N \bredpar N'$ and $Q \bredpar Q'$ then
      $(\lambd[x][N])Q \bredpar \Subst{N'}{Q'}{x}$.
  \end{enumerate}
\end{defn}

Informal interpretation of the above definition: Parallel reduction allows us to reduce any number of redices in the
term in one step. It's different from reduction in the sense that we
can only contract redices that's in the original term, but not redices
arising in current parallel reduction. For example, the term
$(\lambd[f][fx])(\lambd[y][y])$ can be only be parallel reduced
to itself or $(\lambd[y][y])x$, but not further to $x$ (reduction can do
that), because this redex arises in this step of parallel reduction,
so we can't contract it. However we can perform another parallel
reduction      to get it.

\begin{thm}\ollabel{thm:refl}
  $M \bredpar M$.
\end{thm}
\begin{proof}
  Left as exercise, you only need the first three rules.
\end{proof}

\begin{defn}[$\beta$-complete development]\ollabel{bcd}
  The \emph{$\beta$-complete development} of $M$, or $\bcd{M}$, is defined
  as follows:
  \begin{align}
    \bcd{x} &= x \ollabel{bcd:1} \\
    \bcd{\lambd[x][N]} &= \lambd[x][\bcd{N}] \ollabel{bcd:2}\\
    \bcd{PQ} &= \bcd{P}\bcd{Q} && \text{if $P$ is not abstraction} \ollabel{bcd:3} \\
    \bcd{(\lambd[x][N])Q} &= \Subst{\bcd{N}}{\bcd{Q}}{x} \ollabel{bcd:4}
  \end{align}
\end{defn}

Complete development, as its name suggests, is a ``
complete parallel reduction''. While for parallel reduction we still can choose
to not contract a redex, for complete development we have no choice
but to contract all. Thus the complete
development of $(\lambd[f][fx])(\lambd[y][y])$ is
$(\lambd[y][y])x$, not itself.

\begin{editorial}
  This definition has the problem that we haven't introduced how to
  define functions on ($\Lambda$-)terms recursively. Will fix in future.
\end{editorial}

\begin{lem}\ollabel{lem:comp}
  If $M \bredpar M'$ and $R \bredpar R'$, then $\Subst{M}{P}{y}
  \bredpar \Subst{M'}{P'}{y}$.
\end{lem}
\begin{proof}
  By induction on the derivation of $M \bredpar M'$.
  \begin{enumerate}
    \item If the last step is \olref{def:1}, left as exercise.
    \item If the last step is \olref{def:2}, then $M$ is 
      $\lambd[x][N]$ and $M'$ is $\lambd[x][N']$,
      where $N \bredpar N'$. What we want to prove is
      $\Subst{(\lambd[x][N])}{R}{y} \bredpar
      \Subst{(\lambd[x][N'])}{R'}{y}$, which is
      $\lambd[x][\Subst{N}{R}{y}] \bredpar
      \lambd[x][\Subst{N'}{R}{y}]$, which is immediate by
      \olref{def:2} and I.H.
    \item If the last step is \olref{def:3}, left as exercise.
    \item If the last step is \olref{def:4}, then $M$ is
      $(\lambd[x][N])Q$ and $M'$ is $\Subst{N'}{Q'}{x}$. What
      we want to prove is $\Subst{((\lambd[x][N])Q)}{R}{y}
      \bredpar \Subst{\Subst{N'}{Q'}{x}}{R'}{y}$, which is
      $(\lambd[x][\Subst{N}{R}{y}])\Subst{Q}{R}{y} \bredpar
      \Subst{\Subst{N'}{R'}{y}}{\Subst{Q'}{R'}{y}}{x}$, which
      can be derived by \olref{def:4} with I.H.
  \end{enumerate}
\end{proof}

\begin{lem}\ollabel{lem:cont}
  If $M \bredpar M'$ then $M' \bredpar \bcd{M}$.
\end{lem}
\begin{proof}
  By induction on the derivation of $M \bredpar M'$.
  \begin{enumerate}
    \item If the last rule is \olref{def:1}, left as exercise.
    \item If the last rule is \olref{def:2} then $M$ is 
      $\lambd[x][N]$ and $M'$ is $\lambd[x][N']$ with
      $N \bredpar N'$. We want $\lambd[x][N'] \bredpar
      \bcd{(\lambd[x][N])}$, which is $\lambd[x][N'] \bredpar
      \lambd[x][\bcd{N}]$ by \olref{bcd:2}, which is immediate by
      \olref{def:2} and I.H.. 
    \item If the last rule is \olref{def:3} then $M$ is 
      $PQ$ and $M'$ is $P'Q'$ for some $P$, $Q$, $P'$ and $Q'$, with $P \bredpar P'$
      and $Q \bredpar Q'$. By I.H. we have $P' \bredpar \bcd{P}$ and
      $Q' \bredpar \bcd{Q}$.  Depending on whether $P$ is 
      $\lambd[x][N]$ for some $x$ and $N$:
      \begin{enumerate}
        \item $P$ is not, then $P'Q' \bredpar \bcd{P}\bcd{Q}$ by
          \olref{def:3}, and the right-hand is $\bcd{PQ}$ by \olref{bcd:3}.
        \item $P$ is $\lambd[x][N]$ for some $x$ and $N$, then
          $P'$ must be $\lambd[x][N']$ for some $N'$ with 
          $N \bredpar N'$. By I.H. we have $N' \bredpar \bcd{N}$ and
          $Q' \bredpar \bcd{Q}$, then $(\lambd[x][N'])Q' \bredpar
          \Subst{\bcd{N}}{\bcd{Q}}{x}$ by \olref{def:4} which is what
          we want.
      \end{enumerate}
    \item If the last rule is \olref{def:4} then $M$ is 
      $(\lambd[x][N])Q$  and $M'$ is $\Subst{N'}{Q'}{x}$
      for some $x, N, Q, N', Q'$, with $N \bredpar N'$ and $Q
      \bredpar Q'$. By I.H. we know $N' \bredpar \bcd{N}$ and $Q'
      \bredpar \bcd{Q}$, further by \olref{lem:comp} we have
      $\Subst{N'}{Q'}{x} \bredpar \Subst{\bcd{N}}{\bcd{Q}}{x}$, the
      right hand of which is exactly $\bcd{((\lambd[x][N])Q)}$.
  \end{enumerate}
\end{proof}

\begin{thm}\ollabel{thm:cr}
  $\bredpar$ has Church-Rosser property.
\end{thm}
\begin{proof}
  Immediate from \olref{lem:cont}.
\end{proof}

\end{document}

