% Part: lambda-calculus
% Chapter: church-rosser
% Section: definitions-and-properties

\documentclass[../../../include/open-logic-section]{subfiles}

\begin{document}

\olfileid{lc}{cr}{dap}

\olsection{Definition and Properties}

In this chapter we introduce the concept of Church-Rosser property and
some common properties of this property.

\begin{defn}[Church-Rosser property, CR]
  A relation $\xredone$ on terms is said to satisfy \emph{Church-Rosser
    property} iff:
  
  If $M \xredone P$ and $M \xredone Q$, then there exsits some $N$
  such that $P \xredone N$ and $Q \xredone N$.
\end{defn}

\begin{thm} \ollabel{thm:str}
  If a relation $\xredone$ satisfies Church-Rosser property, and $\xred$ is the
  smallest transitive relation containing $\xredone$, then $\xred$ satisfies
  Church-Rosser property too.
\end{thm}
\begin{proof}
  Suppose $M \xredone P_1 \xredone \ldots \xredone P_m$, $M \xredone Q_1
  \xredone \ldots \xredone Q_n$.
  
  we will prove this theorem by constructing a grid $N$ of terms, whose height is $m + 1$ and width $n + 1$. We use $N_{i,j}$ to denote the terms
  on the $i$-th row and $j$-th column.
  
  We construct $N$ in such a way that $N_{i,j} \xredone N_{i+1,j}$ and
  $N_{i,j} \xredone N_{i,j+1}$. It's defined as follows:
  \begin{align*}
    N_{0,0} &= M \\
    N_{i,0} &= P_i && 1 \le i \le m \\
    N_{0,j} &= Q_j && 1 \le j \le n \\
    otherwise: & \\
    N_{i,j} &= R && N_{i-1,j} \xredone R, N_{i,j-1} \xredone R \\
            & && \text{by CR of $\xredone$}
  \end{align*}

  Now we have $N_{m,0} \xredone \ldots \xredone N_{m,n}$ and $N_{0,n}
  \xredone \ldots \xredone N_{m,n}$. Note $N_{m,0}$ is $P$ and $N_{0,n}$
  is $Q$. By definition of $\xred$ the theorem follows.
\end{proof}

\end{document}
