% Part: lambda-calculus
% Chapter: church-rosser
% Section: beta-reduction

\documentclass[../../../include/open-logic-section]{subfiles}

\begin{document}

\olfileid{lc}{cr}{b}

\olsection{$\beta$-reduction}

In this section we will prove that $\beta$-reduction ($\bred$) also has
Church-Rosser property based on conclusions in last section.

\begin{lem}\ollabel{lem:one-par}
  If $M \bredone M'$, then $M \bredpar M'$.
\end{lem}
\begin{proof} By induction on the derivation of $M \bredone M'$.
  \begin{enumerate}
  \item If the last rule is the contraction, then $M$ is
    $(\lambd[x][N])Q$, $M'$ is $\Subst{N}{Q}{x}$, for some
    $x, N, Q$. Since $N \bredpar N$ and $Q \bredpar Q$ by
    \olref[pb]{thm:refl}, we immediately have $(\lambd[x][N])Q
    \bredpar \Subst{N}{Q}{x}$ by \olref[pb]{def:4}.
  \item The other cases are left as exercises.
  \end{enumerate}
\end{proof}

\begin{lem}\ollabel{lem:par-red}
  If $M \bredpar M'$, then $M \bred M'$.
\end{lem}
\begin{proof} Induction on the derivation of $M \bredpar M'$.
  \begin{enumerate}
  \item If the last rule is \olref[pb]{def:1}, left as exercise.
  \item If the last rule is \olref[pb]{def:2}, then $M$ is 
    $\lambd[x][N]$ and $M'$ is $\lambd[x][N']$ for some $x, N, N'$, where
    $N \bredpar N'$. By I.H. we have $N \bred N'$, then obviously
    $\lambd[x][N] \bred \lambd[x][N']$ (by the same series of
    $\bredone$ as $N \bred N'$).
  \item If the last rule is \olref[pb]{def:3}, then $M$ is 
    $PQ$ and $M'$ is $P'Q'$ for some $P, Q, P', Q'$, where $P \bredpar P'$
    and $Q \bredpar Q'$. Now by I.H. we have $P \bred P'$ and $Q \bred
    Q'$, then $PQ \bred P'Q'$ by the series of $P \bred P'$ followed
    by the series of $Q \bred Q'$.
  \item If the last rule is \olref[pb]{def:4}, then $M$ is
    $(\lambd[x][N])Q$ and $M'$ is  $\Subst{N'}{Q'}{x}$ for some
    $x, N, M', Q, Q'$, where $N \bredpar N'$ and $Q \bredpar Q'$. By I.H. we get $Q \bred
    Q'$ and $N \bred N'$, then $(\lambd[x][N])Q \bred
    \Subst{N'}{Q'}{x}$ by the series of $N \bred N'$ followed by the
    series of $Q \bred Q'$ and finally contraction of
    $(\lambd[x][N'])Q'$.
  \end{enumerate}
\end{proof}


\begin{lem}\ollabel{lem:str}
  $\bred$ is the smallest transitive relation containing $\bredpar$.
\end{lem}
\begin{proof}
  Let $\xred$ be the smallest transitive relation containing $\bredpar$.
  Recall that it can be inductively defined by the rules of transitivity, plus the rule ``if
  $M \bredpar N$ then $M \xred N$''. Now we prove $\bred = \xred$:
  
  $\bred \in \xred$. For any $M \bred M'$, we can extract a series of
  $\bredone$ from it and, by \olref{lem:one-par}, we can change it
  into a series of $\bredpar$, which is $\xred$ by its definition.

  $\xred \in \bred$. For any $M \xred M'$, we can extract a series of
  $\bredpar$ and by \olref{lem:par-red} we can change it into a series
  of $\bred$, which, combined, is another $\bred$.
\end{proof}

\begin{thm}{thm:cr}
  $\bred$ satisfies Church-Rosser property.
\end{thm}
\begin{proof}
  Immediate from \olref[dap]{thm:str}, \olref[pb]{thm:cr} and \olref{lem:str}.
\end{proof}
\end{document}
