% Part: lambda-calculus
% Chapter: church-rosser
% Section: beta-reduction

\documentclass[../../../include/open-logic-section]{subfiles}

\begin{document}

\olfileid{lc}{cr}{bet}

\olsection{$\beta$-reduction}

In this section we will prove $\beta$-reduction ($\red$) also has
Church-Rosser property based on conclusions in last section.

\begin{lem}\ollabel{lem:one-par}
  If $M \redone M'$, then $M \redpar M'$.
\end{lem}
\begin{proof} By induction on the derivation of $M \redone M'$.
  \begin{enumerate}
  \item If the last rule is \olref[int][bet]{def:4}, then $M$ is
    $(\lambd[x][N])Q$, $M'$ is $\Subst{N}{Q}{x}$, for some
    $x, N, Q$. Since $N \redpar N$ and $Q \redpar Q$ by
    \olref[par]{thm:refl}, we immediately have $(\lambd[x][N])Q
    \redpar \Subst{N}{Q}{x}$ by \olref[par]{def:4}.
  \item The other cases are left as exercises.
  \end{enumerate}
\end{proof}

\begin{lem}\ollabel{lem:par-red}
  If $M \redpar M'$, then $M \red M'$.
\end{lem}
\begin{proof} Induction on the derivation of $M \redpar M'$.
  \begin{enumerate}
  \item Left as exercise.
  \item If the last rule is \olref[pr]{def:2}, then $M$ is 
    $\lambd[x][N]$ and $M'$ is $\lambd[x][N']$ for some $x, N, N'$, where
    $N \redpar N'$. By I.H. we have $N \red N'$, then obviously
    $\lambd[x][N] \red \lambd[x][N']$ (by the same series of
    $\redone$ as $N \red N'$).
  \item If the last rule is \olref[pr]{def:3}, then $M$ is 
    $PQ$ and $M'$ is $P'Q'$ for some $P, Q, P', Q'$, where $P \redpar P'$
    and $Q \redpar Q'$. Now by I.H. we have $P \red P'$ and $Q \red
    Q'$, then $PQ \red P'Q'$ by the series of $P \red P'$ followed
    by the series of $Q \red Q'$.
  \item If the last rule is \olref[pr]{def:4}, then $M$ is
    $(\lambd[x][N])Q$ and $M'$ is  $\Subst{N'}{Q'}{x}$ for some
    $x, N, M', Q, Q'$, where $N \redpar N'$ and $Q \redpar Q'$. By I.H. we get $Q \red
    Q'$ and $N \red N'$, then $(\lambd[x][N])Q \red
    \Subst{N'}{Q'}{x}$ by the series of $N \red N'$ followed by the
    series of $Q \red Q'$ and finally contraction of
    $(\lambd[x][N'])Q'$.
  \end{enumerate}
\end{proof}


\begin{thm}[Church-Rosser property of  $\red$]
  $\red$ satisfies Church-Rosser property.
\end{thm}
\begin{proof}
  If $M \red P$ by $m$ steps, and $M \red Q$ by $n$ steps.  Suppose $M \redone P_1 \redone \ldots \redone P_m$, $M \redone Q_1
  \redone \ldots \redone Q_n$. By \olref{lem:one-par} we can replace
  $\redone$ with $\redpar$.
  
  we will prove this theorem by constructing a grid $N$ of terms, whose height is $m + 1$ and width $n + 1$. We use $N_{i,j}$ to denote the terms
  on the $i$-th row and $j$-th column.

  $N_{i,j}$ is defined as follows:
  \begin{align*}
    N_{0,0} &= M \\
    N_{i,0} &= P_i && 1 \le i \le m \\
    N_{0,j} &= Q_j && 1 \le j \le n \\
    otherwise: & \\
    N_{i,j} &= R && N_{i-1,j} \redpar R, N_{i,j-1} \redpar R \\
    & && \text{by \olref[par]{thm:cr}}
  \end{align*}

  Now we have $N_{m,0} \redpar \ldots \redpar N_{m,n}$ and $N_{0,n}
  \redpar \ldots \redpar N_{m,n}$. Note $N_{m,0}$ is $P$ and $N_{0,n}$
  is $Q$, also we can replace $\redpar$ with $\red$ by
  \olref{lem:par-red}. By transitivity of $\red$ the theorem follows.
\end{proof}

\end{document}
