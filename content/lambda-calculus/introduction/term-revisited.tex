% Part: lambda-calculus
% Chapter: introduction
% Section: term-revisited

\documentclass[../../../include/open-logic-section]{subfiles}

\begin{document}

\olfileid{lc}{int}{tr}
\olsection{Term Revisited}

Because the $\alpha$-equivalence comes nearly as granted, we now
identify terms that are $\alpha$-equivalent, as follows:

From now on, when we write a term, what we mean is the
$\alpha$-equivalance class (\olref[sfr][rel][prp]{sec}) it's in. For
example, we write $\lambd[a][\lambd[b][a c]]$ for the
$\alpha$-equivalence class containing terms like
$\lambd[a][\lambd[b][a c]]$, $\lambd[b][\lambd[a][b z]]$, etc.

Also, while in previous sections letters such as $N, Q$ are used to
denote a term, from now on we use them to denote a class,
and it's these classes instead of terms that
will be our subjects of study in the rest of the part. Letters such as
$x, y$ continues to denote a variable.

We also adopt a notation $\rep{M}$ to denote a random member in the
class $M$ (the selection does not matter), and $\rep{M}[0], \rep{M}[1], etc. $ if we need more than
one member.

Besides, we reuse the notations from terms to simplify our wording. We
have following definition on classes:
\begin{defn}
  \begin{enumerate}
  \item $\lambd[x][N]$ is defined to the class containing
    $\lambd[x][\rep{N}]$.
  \item $PQ$ is defined to be the class containing $\rep{P}\rep{Q}$.
  \end{enumerate}
\end{defn}

It's not hard to see that they are well defined, because
$\alpha$-conversion is compatible.

\begin{defn} \ollabel{def:fv}
  The free variable of an $\alpha$-equivalence class $M$, or $FV(M)$, is defined to be $FV(\rep{M})$.
\end{defn}

This definition is well defined because $FV(\rep{M}[0]) = FV(\rep{M}[1])$, as shown in \olref[alp]{thm:fv}.

We also reuse the notation of substition into classes:
\begin{defn} \ollabel{def:sub}
  The substitution of $R$ for $y$ in $M$, or $\Subst{M}{R}{y}$, is
  defined to be $\Subst{\rep{M}}{\rep{R}}{y}$, for any $\rep{M}$ and
  $\rep{R}$ making the substition defined.
\end{defn}

This definition is well defined as shown in \olref[alp]{cor:sub}.

Note how this definition significantly simplifies our reasoning. For
example:
\begin{align}
  &\Subst{\lambd[x][x]}{y}{x} \ollabel{eq:1}\\
  =&\Subst{\lambd[z][z]}{y}{x} \ollabel{eq:2}\\
  =&\lambd[z][\Subst{z}{y}{x}] \\
  =&\lambd[z][z]
\end{align}

\olref{eq:1} is undefined if we still regard it as substitution on
terms; but as mentioned earlier in this section, we now consider a
written term as the classes it's in, so it's actually a substitution
on classes, which is why \olref{eq:2} can happen: we can replace
$\lambd[x][x]$ with $\lambd[z][z]$ because they belong to the same
class, so the meaning is not damaged.

For the same reason, from now on we will assume that the
representatives we choose always satisfies the conditions needed for
substitution. For example, when we see $\Subst{\lambd[x][N]}{R}{y}$,
we will assume the representative $\lambd[x][N]$ is well chosen that
$x \neq y$ and $x \notin FV(R)$ - because otherwise we can choose another
representative satisfying these conditions and it's still the same
class we are reasoning about.

Now we can say goodbye to terms and embrace classes. It's a bit
strange to call $\lambd[x][x]$ a ``class'', let's call them
$\Lambda$-terms (or simply ``terms'' in the rest of the part) from now on,
to distinguish them from $\lambda$-terms that we are familiar with. 

\begin{editorial}
  We cannot say goodbye to terms yet: the whole definition of
  $\Lambda$-terms is based on $\lambda$-terms, and we haven't provided
  a method to define functions on $\Lambda$-terms, which means all
  such functions have to be firstly defined  on $\lambda$-terms, and
  then ``projected'' to $\Lambda$-terms, as we did for substitutions.
  However we assume the reader can intuitively understand how we can
  define functions on $\Lambda$-terms, which is why we will give a
  recursive definition in \olref[cr][pr]{cd}.
\end{editorial}
\end{document}
