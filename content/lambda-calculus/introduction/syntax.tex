% Part: lambda-calculus
% Chapter: introduction
% Section: syntax

\documentclass[../../../include/open-logic-section]{subfiles}

\begin{document}

\olfileid{lc}{int}{syn}
\olsection{Syntax}

\begin{defn}{Terms} \ollabel{def}
The set of \emph{terms} of lambda calculus is defined inductively by:
\begin{enumerate}
  \item \ollabel{def:1} If $x$ is a variable, then $x$ is a
    term called \emph{Variable}.
  \item \ollabel{def:2} If $x$ is a variable and $M$ is a term, then $(\lambd[x][M])$ is
    a term called an \emph{abstraction}. The $x$ in
    $\lambd[x]$ is called a \emph{parameter}, or \emph{binding variable}.
  \item \ollabel{def:3} If both $M$ and $N$ are terms, then
    $(MN)$ is a term, called an \emph{application}.
\end{enumerate}
\end{defn}

We will distinguish meta-variables and variables by typeface. $x, y,
z, etc$ are meta-variables for variables, and $M, N, P, Q, etc$ are
meta-variables for terms. Moreover,  $x, y, z, etc$ are variables
we will use. For example, when we mention $\lambd[x][M]$, it can be
any term starts with abstraction followed by anything, like
$\lambd[x][x x]$ where $x$ is $x$ and $M$ is $x x$.


The terms defined above are fully parenthesized, which can get rather
cumbersome as the term $(\lambd[x][((\lambd[x][x])(\lambd[x][(xx)]))])$ demostrates. A neater way to look at lambda terms is to think of them as
a description of formation. For
example, the last step of forming the term $(\lambd[x][((\lambd[x][x])(\lambd[x][(xx)]))])$
must be abstraction where the !!{parameter} is $x$, before which it is
an application of two terms, the last step of forming both of which is
abstraction, etc.. 

\begin{prob}
  Describe the formation of $(\lambd[g][(\lambd[x][(g (x x))]) (\lambd[x][(g (x x))])])$.
\end{prob}

Intead of plain English, we can make it more intuitive by drawing a
\emph{formation tree}.

\begin{prob}
  Draw the formation tree of $(\lambd[g][(\lambd[x][(g (x x))]) (\lambd[x][(g (x x))])])$.
\end{prob}

We may wonder if for each term there is a unique formation, and there
is. We now formalize this intuition.
\end{document}