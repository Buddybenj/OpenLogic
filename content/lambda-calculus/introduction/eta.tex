% Part: lambda-calculus
% Chapter: introduction
% Section: eta

\documentclass[../../../include/open-logic-section]{subfiles}

\begin{document}

\olfileid{lc}{int}{eta}
\olsection{$\eta$-conversion}

There is another relation on $\lambda$ terms. In
\olref[fv]{sec} we used the example $\lambd[g][\lambd[m][\times g
m]]$. If we look at it carefully it's a function that accepts two~!!{argument}s
$g$ and $m$, and applies $\times$ on them. It seems to be another
multiplication function. We use $\eta$-reduction (and $\eta$-extension) to
capture this idea.

\begin{defn}[$\eta$-contraction, $\eredone$]
  $\eta$-contraction ($\eredone$) is the smallest compatible relation
  on terms satisfying the following condition:
  \begin{align*}
    &\lambd[x][f x] \eredone f
  \end{align*}
\end{defn}

\begin{defn}[$\eta$-reduction, $\ered$]
  $eta$-reduction, $\ered$ is the smallest transitive relation on terms
  containing $eredone$.
\end{defn}

\begin{prob}
  There is a notion of \emph{$\eta$-expansion} kind of symmetric to
  $eta$-reduction, define it.
\end{prob}

\begin{defn}[$\eta$-equivalence, $\eeq$]
  $\eta$-equivalence is the smallest reflexive, symmetric, transitive, compatible relation on terms containing $\eredone$.
\end{defn}

\end{document}
