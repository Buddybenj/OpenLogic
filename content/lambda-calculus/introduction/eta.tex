% Part: lambda-calculus
% Chapter: introduction
% Section: eta

\documentclass[../../../include/open-logic-section]{subfiles}

\begin{document}

\olfileid{lc}{int}{eta}
\olsection{$\eta$-conversion}

There is another relation on $\lambda$ terms. In
\olref[lc][int][fv] we use the example $\lambd[g][\lambd[m][\times g
m]]$. If we look at it carefully it's a function that accepts two~!!{argument}s
$g$ and $m$, and applies $\times$ on them. It seems to be another
multiplying function. We use $\eta$-reduction (and $\eta$-extension) to
capture this idea.

\begin{defn}[$\eta$-reduction]
  If a term $P$ contains an occurrence of $\lambda[x][M x]$, and $x
  \notin FV(M)$, then replacing this occurrence by $M$ resulting in
  $P'$ is called \emph{$\eta$-contraction}. If $P$ can be changed to
  $Q$ by a finite (possibly empty) series of such steps, then we say
  $P ~\alpha\text{-reduces to}~Q$.
\end{defn}

\begin{prob}
  There is a notion of \emph{$\eta$-extension} kind of symmetric to
  $eta$-reduction, define it.
\end{prob}

The smallest transitive relation containing both $\eta$-reduction and
$\eta$-extension is called $\eta$-conversion.

\end{document}
