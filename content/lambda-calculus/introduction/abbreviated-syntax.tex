% Part: lambda-calculus
% Chapter: introduction
% Section: abbreviated-syntax

\documentclass[../../../include/open-logic-section]{subfiles}

\begin{document}

\olfileid{lc}{int}{abb}
\olsection{Abbreviated Syntax}

Since term is just a written version of the formation, and the terms
defined in \olref[lc][int][syn]{def} are cumbersome, we may invent another more concise version, as
long as each of its terms describes a unique formation. One
widely used version called \emph{abbreviated terms} is shown as
follows, in the form of expansion rules from the abbreviated terms to the (cumbersome) terms.

\begin{enumerate}
\item When parentheses are left out, application takes place from left
  to right. For example, if $M$, $N$, $P$, and $Q$ are terms, then
  $MNPQ$ abbreviates $(((MN)P)Q)$.
\item Again, when parentheses are left out, lambda abstraction is to
  be given the widest scope possible. From example, $\lambd[x][MNP]$ is
  read $\lambd[x][(MNP)]$.
\item A lambda can be used to abstract multiple variables. For
  example, $\lambd[xyz][M]$ is short for
  $\lambd[x][\lambd[y][\lambd[z][M]]]$.
\end{enumerate}

For example,
\[
\lambd[xy][xxyx \lambd[z][xz]]
\]
abbreviates
\[
(\lambd[x][(\lambd[y][((((xx)y)x)(\lambd[z][(xz)]))])]).
\]

\begin{prob}
  Expand this abbreviated term: $\lambd[g][(\lambd[x][g (x x)]) \lambd[x][g (x x)]]$.
\end{prob}

\begin{prob}
  Prove the unique readability of abbreviated terms as we did for
  cumbersome terms in \olref[unq]{prop:unq}.
\end{prob}

We can now think of abbreviated terms, cumbersome terms and
formation trees as the same
thing. We will only use the abbreviated terms in the rest of the
text, since it is the most compact, and when we say ``terms'' we mean
abbreviated ones.

\end{document}
