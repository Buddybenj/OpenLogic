% Part: lambda-calculus
% Chapter: introduction
% Section: beta

\documentclass[../../../include/open-logic-section]{subfiles}

\begin{document}

\olfileid{lc}{int}{bet}
\olsection{$\beta$-reduction}

When we see $(\lambd[m][\times g m]) 9.8$, it's natural to
conjecture that it has some connection with $\times g 9.8$,
namely  $\times g 9.8$ should be the result of ``simplifying'' the first term, if
``simplify'' means anything. The $\beta$-reduction captures this
intuition formally.

\begin{defn}[$\beta$-reduction]
  If a term $P$ contains an occurrence of $(\lambd[x][M])N$ (called \emph{redex}), and $\Subst{M}{N}{x}$
  is defined, then replacing this occurrences by 
  \begin{equation*}
    \Subst{M}{N}{x}
  \end{equation*}
  which results in $Q$, is called \emph{contracting the redex}, and we
  say $P ~\beta\text{-contracts to} ~Q$, or $P \redone Q$.  If $P$ can
  be changed to $Q$ by a finite (possibly empty) series of
  $\beta$-contractions and $\alpha$-conversions, then we say $P
  ~\beta\text{-reduces to}~Q$, or $P \red Q$.
\end{defn}

A term that cannot be $\beta$-reduced any further is called \emph{$\beta$-irreducible}, or
$\beta$-normal. We will say ``reduces'' instead of ``$\beta$-reduces,''
etc., when the context is clear.

Let us consider some examples.
\begin{enumerate}
\item We have
\begin{align*}
(\lambd[x][xxy]) \lambd[z][z] & \redone (\lambd[z][z])(\lambd[z][z]) y \\
& \redone (\lambd[z][z]) y \\
& \redone y
\end{align*}
\item ``Simplifying'' a term can make it more complex:
\begin{align*}
(\lambd[x][xxy])(\lambd[x][xxy]) & \redone (\lambd[x][xxy])(\lambd[x][xxy])y \\
& \redone (\lambd[x][xxy])(\lambd[x][xxy])yy \\
& \redone \dots
\end{align*}
\item It can also leave a term unchanged:
\[
(\lambd[x][xx])(\lambd[x][xx]) \redone (\lambd[x][xx])(\lambd[x][xx])
\]
\item Also, some terms can be reduced in more than one way; for
  example,
\[
(\lambd[x][(\lambd[y][yx]) z)] v \redone (\lambd[y][yv]) z
\]
by contracting the outermost application; and
\[
(\lambd[x][(\lambd[y][yx]) z)] v \redone (\lambd[x][zx]) v
\]
by contracting the innermost one. Note, in this case, however, that
both terms further reduce to the same term, $zv$.
\end{enumerate}

The final outcome in the last example is not a coincidence, but rather
illustrates a deep and important property of the lambda calculus, known as the
``Church-Rosser property.''

\end{document}