% Part: lambda-calculus
% Chapter: introduction
% Section: beta

\documentclass[../../../include/open-logic-section]{subfiles}

\begin{document}

\olfileid{lc}{int}{bet}
\olsection{$\beta$-reduction}

When we see $(\lambd[m][\times g m]) 9.8$, it's natural to
conjecture that it has some connection with $\times g 9.8$,
namely  $\times g 9.8$ should be the result of ``simplifying'' the first term, if
``simplify'' means anything. The $\beta$-reduction captures this
intuition formally.

\begin{defn}[$\beta$-contraction, $\redone$] \ollabel{def}
  The $\beta$-contraction ($\redone$) is the smallest compatible
  relation on terms, satisfying the following condition:
  \begin{align*}
    (\lambd[x][N])Q \redone \Subst{N}{Q}{x}
  \end{align*}
\end{defn}

\begin{prob}
  Spell out the equivalent induction definitions of $\beta$-contraction as we
  did for Change of bound variable in \olref[alp]{def}.
\end{prob}
  
\begin{defn}[$\beta$-reduction, $\red$] \ollabel{def}
  $\beta$-reduction ($\red$) is the smallest reflexitive, transitive
  relation on terms containing $\redone$.
\end{defn}

\begin{defn}[$beta$-normal]
A term that cannot be $\beta$-contracted any further is called \emph{$\beta$-irreducible}, or
$\beta$-normal. We will say ``reduces'' instead of ``$\beta$-reduces,''
etc., when the context is clear.
\end{defn}

Let us consider some examples.
\begin{enumerate}
\item We have
\begin{align*}
(\lambd[x][xxy]) \lambd[z][z] & \redone (\lambd[z][z])(\lambd[z][z]) y \\
& \redone (\lambd[z][z]) y \\
& \redone y
\end{align*}
\item ``Simplifying'' a term can make it more complex:
\begin{align*}
(\lambd[x][xxy])(\lambd[x][xxy]) & \redone (\lambd[x][xxy])(\lambd[x][xxy])y \\
& \redone (\lambd[x][xxy])(\lambd[x][xxy])yy \\
& \redone \dots
\end{align*}
\item It can also leave a term unchanged:
\[
(\lambd[x][xx])(\lambd[x][xx]) \redone (\lambd[x][xx])(\lambd[x][xx])
\]
\item Also, some terms can be reduced in more than one way; for
  example,
\[
(\lambd[x][(\lambd[y][yx]) z]) v \redone (\lambd[y][yv]) z
\]
by contracting the outermost application; and
\[
(\lambd[x][(\lambd[y][yx]) z]) v \redone (\lambd[x][zx]) v
\]
by contracting the innermost one. Note, in this case, however, that
both terms further reduce to the same term, $zv$.
\end{enumerate}

The final outcome in the last example is not a coincidence, but rather
illustrates a deep and important property of the lambda calculus, known as the
``Church-Rosser property.''

\begin{digress}
  In general cases there is more than one way to $\beta$-reduce a
  term, thus many reduction strategies have been invented, among which
  the most common is \emph{natural strategy}.
  
  Natural strategy always contracts the \emph{left-most}
  redex, where the position of a redex is defined as its starting
  point in the term.

  Natural strategy has a useful property that if a term can be reduced
  to normal form by some strategy iff it can be reduced to normal form in
  natural strategy. In the rest of the text we will use natural
  reduction unless otherwise specified. 
\end{digress}

\end{document}