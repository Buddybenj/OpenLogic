% Part: lambda-calculus
% Chapter: introduction
% Section: substitution

\documentclass[../../../include/open-logic-section]{subfiles}

\begin{document}

\olfileid{lc}{int}{sub}
\olsection{Substitution}

As mentioned earlier, free variables are references to environment variables
, thus it makes sense to actually use a specific value in the place of free
variables. For example, we may want to replace $g$ in
$\lambd[m][\times m g]$ by $9.8$, resulting in $\lambd[m][\times m
9.8]$ to actually use the function on earth. This is
called substitution.

We define the subtitution as follows:

\begin{defn}[Substitution]
  The substitution of term $@N$ for a variable $@x$ in a term $@M$, or
  $\Subst{@M}{@N}{@x}$,  is defined inductively by:
  \begin{align}
    \Subst{@x}{@N}{@x}       &= @N \ollabel{eq:1} \\
    \Subst{@y}{@N}{@x}       &= @y, \text{if} @x \neq @y \ollabel{eq:2} \\
    \Subst{@P@Q}{@N}{@x} &= (\Subst{@P}{@N}{@x}) (\Subst{@Q}{@N}{@x}) \ollabel{eq:3} \\
    \Subst{(\lambd[@x][@P])}{@N}{@x}  &= \lambd[@x][@P] \ollabel{eq:4} \\
    \Subst{(\lambd[@y][@P])}{@N}{@x}  &= \lambd[@y][\Subst{@P}{@N}{@x}], \text{if} @x \neq @y,
                                        \text{provided} @y \notin FV(@N) \text{or} @x \notin FV(@P) \ollabel{eq:5}
  \end{align}
  otherwise it's undefined.
\end{defn}

\eqref{eq:1} through~\eqref{eq:3} are routine. The last two equations are
the meat. When we say ``let $m$ be $5$'' the term
$\lambda[m][\times g m]$ won't be affected for a obvious reason, which
is~\eqref{eq:4} all about. 

On the other hand if we say ``let $g$ be $9.8$'' then we actually need
to replace the $g$ in $\lambd[m][\times g m]$ by $9.8$. Note there are
some conditions only under which
\eqref{eq:5} can be applied. Why $@y \notin FV(@N)$? Because in this
case $@N$ contains references to the environment variable $@y$, which,
after $@N$ replaces $@x$ in $@P$, will be bound by the abstraction over
$@y$. That will result in a term with different meaning (if there is any
meaning), as these $@y$ no longer refer to the environment variables but instead the argument
of that abstraction. 

For example, we cannot do $\Subst{\lambd[y][x]}{y}{x}$ because
otherwise it results in $\lambd[y][y]$, a term accepts an argument and
returns it directly. But the substitute $y$ is not likely refering
to the $y$ bound by $\lambd[y]$, but instead, more likely another
environment variable named $y$! So the result we actually want is a
function that accepts an argument, drop it, and returns the
environment variable $y$ anyway.

Now what about $@x \notin FV(@P)$? If there is no free occurrences
of $@x$ in $@P$, then there is nothing to replace, the substitution does
nothing and it's fine even if $@y \in FV(@N)$.

\begin{prob}
  Do the following substitution.
  \begin{enumerate}
  \item $\Subst{\lambd[y][x(\lambd[w][vwx])]}{(uv)}{x}$
  \item $\Subst{\lambd[y][x(\lambd[x][x])]}{(\lambd[y][xy])}{x}$
  \item $\Subst{y(\lambd[v][xv])}{(\lambd[y][vy])}{x}$
  \end{enumerate}
\end{prob}

\begin{thm}\ollabel{thm:notinfv}
  If $@x \notin FV(@M)$, then $\Subst{@M}{@S}{@x}$ is defined and identical to $@M$.
\end{thm}
\begin{proof}
  By induction on the formation of $@M$.
  \begin{enumerate}
  \item[\rule{VAR}] $@M$ is just a variable $@y$. Since $@x \notin
    FV(@M)=FV(@y)=\{@y\}$ we know $@x \neq @y$. Thus we apply
    \eqref{eq:2} and get $@y$ unchanged.
  \item[\rule{ABS}] $@M$ is of the form $\lambd[@y][@N]$. 
    \begin{enumerate}
    \item If $@x \eq @y$ then we apply \eqref{eq:4} and get $\lambd[@y][@N]$
      unchanged.
    \item If $@x \neq @y$ then $@x \notin FV(@N)$ by
      \olref[fv]{eq:2}, then by I.H. we get $\Subst{@N}{@S}{@x}$
      is defined and is identical to $@N$. We now can apply
      \eqref{eq:5} and get $\lambd[@y][\Subst{@N}{@S}{@x}]$, which
      is $@N$.
    \end{enumerate}
  \item[\rule{APP}] $@M$ is of the form $@P@Q$, then $@x \notin
    FV(P)$ and $@x \notin FV(Q)$ by \olref[fv]{eq:3}. We then
    apply \eqref{eq:3} and get
    $(\Subst{@P}{@S}{@x})(\Subst{@Q}{@S}{@x})$, where
    $\Subst{@P}{@S}{@x}$ is defined and identical to $@P$ by I.H.,
    and similar for $\Subst{@Q}{@S}{@x}$. Thus it's identical to $PQ$.
  \end{enumerate}
\end{proof}

\begin{thm}\ollabel{thm:clr}
  $@x \notin FV(\Subst{@M}{@y}{@x})$, if $\Subst{@M}{@y}{@x}$ is defined.
\end{thm}
\begin{proof}
  Left as exercise.
\end{proof}

\begin{thm}\ollabel{thm:inv}
  If $\Subst{@M}{@y}{@x}$ is defined and $@y \notin FV(@M)$, then $\Subst{\Subst{@M}{@y}{@x}}{@x}{@y} \eq @M$.
\end{thm}
\begin{proof}
  By induction on formation of $@M$.
  \begin{enumerate}
    \item[\rule{VAR}] $@M$ is just a variable $@z$.
      \begin{enumerate}
        \item[$@z \eq @x$] Then $\Subst{\Subst{@x}{@y}{@x}}{@x}{@y}$
          equals $\Subst{@y}{@x}{@y}$ equals $@x$ equals $@z$.
        \item[$@z \neq @x$] Then $\Subst{\Subst{z}{@y}{@x}}{@x}{@y}$
          equals $\Subst{@z}{@x}{@y}$. We know $@z \neq @y$ since $@y
          \notin FV(@z)$. So RHS equals $@z$.
      \end{enumerate}
    \item[\rule{ABS}] $@M$ is of the form $\lambd[@z][@N]$.
      \begin{enumerate}
        \item[$@z \eq @x$] Then $\Subst{\Subst{(\lambd[@x][@N])}{@y}{@x}}{@x}{@y}$
          equals $\Subst{(\lambd[@x][@N])}{@x}{@y}$. Depending on
          whether or not $x=y$:
          \begin{enumerate}
            \item[$@x = @y$] Then RHS equals
              $\Subst{(\lambd[@x][@N])}{@x}{@x}$ equals $\lambd[@x][@N]$.
            \item[$@x \neq @y$] 
              RHS equals $\lambd[@x][\Subst{@N}{@x}{@y}]$ since $@y
              \notin FV(@N)$ by \olref[fv]{lem:abs}; equals
              $\lambd[@x][@N]$ by \olref[fv]{lem:notinfv}.
          \end{enumerate}
        \item[$@z \neq @x$] then $\Subst{\Subst{(\lambd[@z][@N])}{@y}{@x}}{@x}{@y}$ equals 
          $\Subst{(\lambd[@z][\Subst{@N}{@y}{@x}])}{@x}{@y}$ 
          simply becuase $\Subst{(\lambd[@z][@N])}{@y}{@x}$ is
          defined (by premise) and 
            $\lambd[@z][\Subst{@N}{@y}{@x}]$ is the only possible result.
            Further we know that $\Subst{@N}{@y}{@x}$ is defined. Now
            we consider whether or not $@y = @z$.
            \begin{enumerate}
              \item[$@y = @z$] Now $@z \in FV(@y)$ so it must be the
                case that $@x \notin FV(@N)$. So it equals
                $\Subst{(\lambd[@z][@N])}{@x}{@y}$ equals
                $\lambd[@z][@N]$.
              \item[$@y \neq @z$] Then it equals
                $\lambd[@z][\Subst{\Subst{@N}{@y}{@x}}{@x}{@y}]$ since
                $@z \notin FV(@x)$. Now by I.H., since
                $\Subst{@N}{@y}{@x}$ is defined, and $@y \notin
                FV(@N)$ by \olref[fv]{abs}, we have
                $\Subst{\Subst{@N}{@y}{@x}}{@x}{@y}$ equals $@N$, so
                the origin expression equals $\lambd[@z][@N]$.
              \end{enumerate} 
      \end{enumerate}
      \item[\rule{APP}] $@M$ is of the form $@P@Q$. Then
        $\Subst{\Subst{(@P@Q)}{@y}{@x}}{@x}{@y}$ equals
        $\Subst{((\Subst{@P}{@y}{@x})(\Subst{@Q}{@y}{@x}))}{@x}{@y}$ equals
        $(\Subst{\Subst{@P}{@y}{@x}}{@x}{@y})(\Subst{\Subst{@Q}{@y}{@x}}{@x}{@y})$.
        We also know that $\Subst{@P}{@y}{@x}$ is defined and $@y
        \notin FV(@P)$ by \olref[fv][app], so by I.H. we get
        $\Subst{\Subst{@P}{@y}{@x}}{@x}{@y}$ equals $@P$ and similar
        for $@Q$. So finally it equals $@P@Q$.
  \end{enumerate}
\end{proof}

\end{document}
