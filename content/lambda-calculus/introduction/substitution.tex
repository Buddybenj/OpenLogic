% Part: lambda-calculus
% Chapter: introduction
% Section: substitution

\documentclass[../../../include/open-logic-section]{subfiles}

\begin{document}

\olfileid{lc}{int}{sub}
\olsection{Substitution}

As mentioned earlier, free variables are references to environment variables
, thus it makes sense to actually use a specific value in the place of free
variables. For example, we may want to replace $g$ in
$\lambd[m][\times m g]$ by $9.8$, resulting in $\lambd[m][\times m
9.8]$ to actually use the function on earth. This is
called substitution.

We define the subtitution as follows:

\begin{defn}[Substitution]
  The substitution of term $@N$ for a variable $@x$ in a term $@M$, or
  $\Subst{@M}{@N}{@x}$,  is defined inductively by:
  \begin{align}
    \Subst{@x}{@N}{@x}       &= @N \label{eq:1} \\
    \Subst{@y}{@N}{@x}       &= @y, \text{if} @x \neq @y \label{eq:2} \\
    \Subst{@P@Q}{@N}{@x} &= (\Subst{@P}{@N}{@x}) (\Subst{@Q}{@N}{@x}) \label{eq:3} \\
    \Subst{(\lambd[@x][@P])}{@N}{@x}  &= \lambd[@x][@P] \label{eq:4} \\
    \Subst{(\lambd[@y][@P])}{@N}{@x}  &= \lambd[@y][\Subst{@P}{@N}{@x}], \text{if} @x \neq @y,
    \text{provided} @y \notin FV(@N) \text{or} @x \notin FV(@P) \label{eq:5}
  \end{align}
  otherwise it's undefined.
\end{defn}

\eqref{eq:1} through~\eqref{eq:3} are routine. The last two equations are
the meat. When we say ``let $m$ be $5$'' the term
$\lambda[m][\times g m]$ won't be affected for a obvious reason, which
is~\eqref{eq:4} all about. 

On the other hand if we say ``let $g$ be $9.8$'' then we actually need
to replace the $g$ in $\lambd[m][\times g m]$ by $9.8$. Note there are
some conditions only under which
\eqref{eq:5} can be applied. Why $@y \notin FV(@N)$? Because in this
case $@N$ contains references to the environment variable $@y$, which,
after $@N$ replaces $@x$ in $@P$, will be bound by the abstraction over
$@y$. That will result in a term with different meaning (if there is any
meaning), as these $@y$ no longer refer to the environment variables but instead the argument
of that abstraction. 

For example, we cannot do $\Subst{\lambd[y][x]}{y}{x}$ because
otherwise it results in $\lambd[y][y]$, a term accepts an argument and
returns it directly. But the substitute $y$ is not likely refering
to the $y$ bound by $\lambd[y]$, but instead, more likely another
environment variable named $y$! So the result we actually want is a
function that accepts an argument, drop it, and returns the
environment variable $y$ anyway.

Now what about $@x \notin FV(@P)$? If there is no free occurrences
of $@x$ in $@P$, then there is nothing to replace, the substitution does
nothing and it's fine even if $@y \in FV(@N)$.

\begin{prob}
  Do the following substitution.
  \begin{enumerate}
    \item $\Subst{\lambd[y][x(\lambd[w][vwx])]}{(uv)}{x}$
    \item $\Subst{\lambd[y][x(\lambd[x][x])]}{(\lambd[y][xy])}{x}$
    \item $\Subst{y(\lambd[v][xv])}{(\lambd[y][vy])}{x}$
  \end{enumerate}
\end{prob}

\end{document}
