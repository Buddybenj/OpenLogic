% Part: lambda-calculus
% Chapter: introduction
% Section: alpha

\documentclass[../../../include/open-logic-section]{subfiles}

\begin{document}

\olfileid{lc}{int}{alp}
\olsection{$\alpha$-conversion}

\begin{defn}[$\alpha$-conversion]
  If a term contains an occurrences of $\lambd[x][M]$, $y \notin
  FV(M)$, and $M[y/x]$ is defined, then replacing this occurrences
  by 
  \begin{equation*}
    \lambd[y][M[y/x]]
  \end{equation*}
  is called a \emph{change of bound
    variable}. If $P$ can be changed to $Q$ by a finite (possibly
  empty) series of such steps, then we say $P ~\alpha\text{-converts to}~ Q$,
  or $P \aconv Q$.
\end{defn}

\begin{ex}
  $\lambd[m][\times m g]$ $\alpha$-converts to $\lambd[M][\times M
  g]$, and the other way around too. Informally
  speaking, they are both functions that accpets an argument and
  returns the gravity applied on the object with mass of that argument,
  refering to the free variable $g$.
\end{ex}
\begin{ex}
  $\lambd[m][\times m g]$ cannot $\alpha$-converts $\lambd[m][\times m
  G]$. Informally speaking, they refers to the environment variables $g$ and $G$ respectively,
  making them different functions: they behave differently in
  environment where $g$ and $G$ is different.
\end{ex}

\begin{prob}
  Tell if the following pairs of terms are $\alpha$-convertable.
  \begin{enumerate}
  \item $\lambd[x][\lambd[y][x]]$ and $\lambd[y][\lambd[x][y]]$
  \item $\lambd[x][\lambd[y][x]]$ and $\lambd[c][\lambd[b][a]]$
  \item $\lambd[x][\lambd[y][x]]$ and $\lambd[c][\lambd[b][a]]$
  \end{enumerate}
\end{prob}

\end{document}

