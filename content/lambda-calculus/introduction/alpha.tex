% Part: lambda-calculus
% Chapter: introduction
% Section: alpha

\documentclass[../../../include/open-logic-section]{subfiles}

\begin{document}

\olfileid{lc}{int}{alp}
\olsection{$\alpha$-conversion}

What's the relation between $\lambd[x][x]$ and $\lambd[y][y]$? They
both seem to be the identity function. Here is the $\alpha$-conversion:

\begin{defn}[Change of bound variable, $\aconvone$]
  If a term $M$ contains an occurrences of $\lambd[x][N]$, $y \notin
  FV(N)$, and $N[y/x]$ is defined, then replacing this occurrences
  by 
  \begin{equation*}
    \lambd[y][N[y/x]]
  \end{equation*}
  resulting in $M'$ is called a \emph{change of bound variable}, written
  as $M \redone[\alpha] M'$.
\end{defn}


We used the word ``occurrence'' in above definition, which is not very
standard but captures the idea. It's mostly known as
\emph{compatibility}. 
\begin{defn}[Compatibility of relation]
  A relation $R$ on terms is said to be \emph{compatible}
  if it satisfies following conditions:
  \begin{enumerate}
  \item If $R N N'$ then $R \lambd[x][N] \lambd[x][N']$
  \item If $R P P'$ then $R (PQ) (P'Q)$
  \item If $R Q Q'$ then $R (PQ) (PQ')$
  \end{enumerate}
\end{defn}

Thus let's rephrase the definition:
\begin{defn}[Change of bound variable, $\aconvone$]
  \emph{Change of bound variable} ($\redone[\alpha]$) is 
  the smalllest compatible relation on terms satisfying following
  condition:
  \begin{align*}
    &\lambd[x][N] \redone[\alpha] \lambd[y][\Subst{N}{y}{x}] \text{if
      $x \neq y$, $y \notin FV(N)$} \\
    & &&\text{and $\Subst{N}{y}{x}$ is defined}
  \end{align*}
\end{defn}

``smallest'' means the relation contains only pairs that's required by
the compatibility and the additional condition, and nothing else. Thus
this relation can also be defined as follows:
\begin{defn}[Change of bound variable, $\aconvone$] \ollabel{def}
  \emph{Change of bound variable} ($\aconvone$) is inductively
  defined as follows:
  \begin{enumerate}
  \item If $N \aconvone N'$ then $\lambd[x][N] \aconvone
    \lambd[x][N']$ \ollabel{def:1}
  \item If $P \aconvone P'$ then $(PQ) \aconvone (P'Q)$ \ollabel{def:2}
  \item If $Q \aconvone Q'$ then $(PQ) \aconvone (PQ')$ \ollabel{def:3}
  \item if $x \neq y$, $y \notin FV(N)$ and $\Subst{N}{y}{x}$ is defined, then
    $\lambd[x][N] \redone[\alpha] \lambd[y][\Subst{N}{y}{x}]$.
    \ollabel{def:4}
    
  \end{enumerate}
\end{defn}

The proof of the equivalence of the two definition is left to
interested readers, but from now on we will use the later one as our
definition.

\begin{defn}[$\alpha$-conversion, $\aconv$]
  \emph{$\alpha$-conversion} ($\aconv$) is the smallest reflexitive, transitive relation on terms containing $\aconvone$.
\end{defn}

As above, ``smallest'' means the relation only contains pairs required
by the transitivity, and $\aconvone$, which leads to the following equivalent definition:
\begin{defn}[$\alpha$-conversion, $\aconv$]
  \emph{$\alpha$-conversion} ($\aconv$) is inductively defined as follows:
  \begin{enumerate}
  \item If $P \aconv Q$ and $Q \aconv R$, then $P \aconv R$.
    \ollabel{aconv:1}
  \item If $P \aconvone Q$, then $P \aconv Q$. \ollabel{aconv:2}
  \item $P \aconv P$. \ollabel{aconv:3}
  \end{enumerate}
\end{defn}

\begin{ex}
  $\lambd[m][\times m g]$ $\alpha$-converts to $\lambd[mass][\times mass
  g]$, and the other way around too. Informally
  speaking, they are both functions that accpets an argument and
  returns the gravity applied on the object with mass of that argument,
  refering to the free variable $g$.
\end{ex}
\begin{ex}
  $\lambd[m][\times m g]$ cannot $\alpha$-converts $\lambd[m][\times m
  G]$. Informally speaking, they refers to the environment variables $g$ and $G$ respectively,
  making them different functions: they behave differently in
  environment where $g$ and $G$ is different.
\end{ex}

\begin{prob}
  Tell if the following pairs of terms are $\alpha$-convertable.
  \begin{enumerate}
  \item $\lambd[x][\lambd[y][x]]$ and $\lambd[y][\lambd[x][y]]$
  \item $\lambd[x][\lambd[y][x]]$ and $\lambd[c][\lambd[b][a]]$
  \item $\lambd[x][\lambd[y][x]]$ and $\lambd[c][\lambd[b][a]]$
  \end{enumerate}
\end{prob}

\begin{lem}\ollabel{lem:fv-one}
  If $P \aconvone Q$ then $FV(P) = FV(Q)$.
\end{lem}
\begin{proof}
  Induction on the derivation of $P \aconvone Q$.
  \begin{enumerate}
  \item If the last rule is \olref{def:4}, then $P$ is of the form
    $\lambd[x][N]$ and $Q$ of the form
    $\lambd[y][\Subst{N}{y}{x}]$, with $x \neq y$, $y \notin
    FV(N)$ and $\Subst{N}{y}{x}$ defined. Depending on whether
    $x \in FV(N)$:
    \begin{enumerate}
    \item $x \in FV(N)$, then:
      \begin{align*}
        &FV(\lambd[y][\Subst{N}{y}{x}])\\
        =&FV(\Subst{N}{y}{x}) \setminus \{y\} \\
        =&((FV(N) \setminus \{x\}) \cup \{y\}) \setminus \{y\}
         && \text{ by \olref[sub]{thm:infv}} \\
        =&FV(N) \setminus \{x\} \\
        =&FV(\lambd[x][N])
      \end{align*}
    \item $x \notin FV(N)$ then:
      \begin{align*}
        &FV(\lambd[y][\Subst{N}{y}{x}])\\
        =&FV(\Subst{N}{y}{x}) \setminus \{y\} \\
        =&FV(N) \setminus \{x\}
         && \text{by \olref[sub]{thm:notinfv}} \\
        =&FV(\lambd[x][N])
      \end{align*}
    \end{enumerate}
  \item The other three cases are left as exercises. \ollabel{lem:fv-one:2}.
  \end{enumerate}
\end{proof}

\begin{prob}
  Finish \olref{lem:fv-one:2}.
\end{prob}

\begin{lem}\ollabel{lem:inv}
  If $P \aconvone Q$ then $Q \aconvone P$.
\end{lem}
\begin{proof}
  Induction on the derivation of $P \redone{\alpha} Q$.
  \begin{enumerate}
  \item Last rule is \olref{def:4}, then $P$ is of the form
    $\lambd[x][N]$ and $Q$ of the form
    $\lambd[y][\Subst{N}{y}{x}]$, where $x \neq y$, $y \notin FV(N)$ and
    $\Subst{N}{y}{x}$ defined. Firstly we have $y \notin
    FV(\Subst{N}{y}{x})$ by \olref[sub]{thm:clr}; Also 
    by \olref[sub]{thm:inv} we have
    $\Subst{\Subst{N}{y}{x}}{x}{y}$ is not only defined, but
    also equal to $N$. Then by \olref{def:4} we have
    $\lambd[y][\Subst{N}{y}{x}] \aconvone
    \lambd[x][\Subst{\Subst{N}{y}{x}}{x}{y}] = \lambd[x][N]$.
  \item The other three cases are left as exercises. \ollabel{lem:inv:2}
  \end{enumerate}
\end{proof}

\begin{prob}
  Prove \olref{lem:inv:2}.
\end{prob}

\begin{thm}
  $\alpha$-conversion is an equivalence relation (\olref[sfr][rel][prp]{sec}) on terms.
\end{thm}
\begin{proof}
  \begin{enumerate}
  \item[reflexive] For each term $M$, $M$ can be changed to $M$ by
    \emph{zero} steps of change of bound variables.
  \item[symmetric] If $P$ is changed to $Q$ by a series of change
    of bound variables, then from $Q$ we can just inverse these
    changes (by \olref{lem:inv}) in
    opposite order and we should get $P$.
  \item[transitive] If $P$ is changed to $Q$ by a series of
    change, and $Q$ to $R$ by another series, then we can change
    $P$ to $R$ by first applying the first series and then the
    second series.
  \end{enumerate}
\end{proof}

From now on we write $M \aeq N$ if $M$ is $\alpha$-equivalent to
$N$, although there is no difference between $\alpha$-equivalence and
$\alpha$-conversion.

\begin{thm}\ollabel{thm:fv}
  If $M \aeq N$, then $FV(M) = FV(N)$.
\end{thm}
\begin{proof}
  Immediate from \olref{lem:fv-one}.
\end{proof}

\begin{lem}\ollabel{lem:sub:R}
  If $R \aeq R'$ and $\Subst{M}{R}{y}$ is defined, then $\Subst{M}{R'}{y}$ is
  defined and $\alpha$-equivalent to $\Subst{M}{R}{y}$.
\end{lem}
\begin{proof}
  Left as exercise.
\end{proof}

\begin{prob}
  Prove \olref{lem:sub:R}.
\end{prob}

Recall that in \olref[sub]{sec} substitution is
undefined in some cases; however, by some
$\alpha$-conversion on terms, we can make the substitution defined,
and the results preserve the $\alpha$-equivalence, as shown in this theorem:
\begin{thm}\ollabel{thm:sub}
  For any $M$, for any $R, y$, there exists $M'$
  such that $M \aeq M'$ and $\Subst{M'}{R}{y}$ is defined; moreover, if there's another pair of
  $M''$ and $R''$ satisfying such condition and $R'' \aeq R$, then $\Subst{M'}{R}{y} \aeq \Subst{M''}{R''}{y}$.
\end{thm}
\begin{proof}
  By induction on the formation of $M$:
  \begin{enumerate}
  \item $M$ is of the form $\lambd[x][N]$, then we:
    \begin{enumerate}
    \item Select $z$ such that $z \notin FV(xyNR)$.
    \item By I.H., we can have $N'$ such that $N' \aeq N$
      and $\Subst{N'}{z}{x}$ is defined. Then $\lambd[x][N] \aeq
      \lambd[x][N']$ too, by \olref{def:1}.
    \item Now $\lambd[x][N'] \aeq \lambd[z][\Subst{N'}{z}{x}]$ by \olref{def:4}.
      we can do this becuase $z \ne x$, $z \notin FV(N')$ and
      $\Subst{N'}{z}{x}$ is defined.
    \item Finally we can see that
      $\Subst{\lambd[z][\Subst{N'}{z}{x}]}{R}{y}$ is defined,
      because $z \neq y$ and $z \notin FV(R)$.
    \item Moreover, if there's another $N''$ and $R''$ satisfying such
      property, then:
      \begin{align*}
        &\Subst{(\lambd[z][\Subst{N''}{z}{x}])}{R''}{y}\\
        =&\lambd[z][\Subst{\Subst{N''}{z}{x}}{R''}{y}] \\
        =&\lambd[z][\Subst{\Subst{N''}{z}{x}}{R}{y}]
         && \text{by \olref{lem:sub:R}}\\
        =&\lambd[z][\Subst{\Subst{N'}{z}{x}}{R}{y}]
         && \text{by I.H.}\\
        =&\Subst{(\lambd[z][\Subst{N'}{z}{x}])}{R}{y}
      \end{align*}
    \end{enumerate}
  \item The other two cases are left as
    exercises. \ollabel{thm:sub:2}
  \end{enumerate}
\end{proof}

\begin{prob}
  Prove \olref{thm:sub:2}.
\end{prob}

\begin{cor}\ollabel{cor:sub}
  For any $M$, for any $R, y$, there exists a pair of $M'$ and $R'$
  such that $M \aeq M'$, $R \aeq R'$ and $\Subst{M'}{R'}{y}$ is defined; moreover,
  if there's another pair of
  $M''$ and $R''$ satisfying such condition, then $\Subst{M'}{R'}{y} \aeq \Subst{M''}{R''}{y}$.
\end{cor}
\begin{proof}
  Immediate from \olref{thm:sub}.
\end{proof}

\begin{digress}
  Because the $\alpha$-equivalence comes so naturally, some syntax has
  been invented that doesn't distinguish terms that can be
  $\alpha$-converted to each other, among which the most known is the
  \emph{De Bruijn index}.
  
  When we write $\lambd[x][M]$, we explicitly state that $x$ is the
  parameter of the function, so that we can use $x$ in $M$ to refer
  to this parameter. In De Bruijn index, however, parameters have no
  name and reference to them in function body is denoted by the number of levels of
  abstraction between them. For example, instead of $\lambd[x][lambd[y][y
  x]]$, we write $\lambd[][lambd[][0 1]]$, since there is zero
  abstraction between the variable $y$ and the abstraction where the
  parameter $y$ is stated, while $x$ is $1$ since there is one level
  of abstraction (namely the abstraction of $y$) between.

  The De Bruijn index is almost unreadable to human, but it's
  much more convenient when we are implementing lambda calculus in
  computer.
\end{digress}

Besides the De Bruijn index, we can also utilize $\alpha$-equivalence
by $\alpha$-equivalence class (\olref[sfr][rel][prp]{sec}), and this is
the approach we will see in next section and use in rest of the part.
\end{document}

