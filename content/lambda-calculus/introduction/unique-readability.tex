% Part: lambda-calculus
% Chapter: introduction
% Section: unique-readability

\documentclass[../../../include/open-logic-section]{subfiles}

\begin{document}

\olfileid{lc}{int}{unq}
\olsection{Unique Readability}

As in \olref[fol][syn][unq], for each lambda term there is only one
way to construct and interpret it. The most of the part makes
sense only if this property holds.

In the following discussion, a \emph{formation} is the procedure of
constructing a term using formation rules (one or several times), \emph{rules} is short for
formation rules. 

\begin{lem}\ollabel{lem:term-start}
  A term starts with either a variable or a parenthesis.
\end{lem}
\begin{proof}
  Left as exercise.
\end{proof}

\begin{lem}\ollabel{lem:app-start}
  The resulting term of APP starts with either two parenthesis or a
  parenthesis and a variable.
\end{lem}
\begin{proof}
  Left as exercise; you may want to use \olref{lem:term-start}.
\end{proof}

\begin{prop}[Unique Readability] \ollabel{prop:unq}
There is a unique formation for each term. In other words, if a term
$M$ is formed by a formation, then it's the only formation that can form this term.
\end{prop}

\begin{proof}
  We prove this by induction on the formation of terms. 

  \begin{enumerate}
    \item[\rule{VAR}] $M$ must be of the form
      $x$, where $x$ is some variable. Since the results of other two
      formations always start with parentheses, they cannot be used to
      construct $M$; Thus, the formation of $M$ must be a single step
      of \rule{VAR}, which is this very formation.
    \item[\rule{ABS}] $M$ must be of the
      form $(\lambd[x][N])$, where $x$ is some variable and $N$ is a
      term constructed from rules. By I.H. we know that formation of
      $N$ is unique. Variable couldn't have construct it
      because the result of that rule is a single variable;
      Neither could \rule{APP} because of \olref{lem:app-start}. Thus
      this formation of $M$, namely the unique formation of $N$
      followed by \rule{ABS}, is unique.

    \item[\rule{APP}] $M$ must be of the form
      $(PQ)$, where $P$ and $Q$ are terms constructed from rules. By
      I.H. we know that formations of $P$ and $Q$ is unique. \rule{VAR} and
      \rule{ABS} couldn't have construct it with the same reason in the 
      last case. Thus this formation of $M$, namely the unique
      formations of $P$ and $Q$ followed by \rule{APP}, is unique.
  \end{enumerate}
\end{proof}

A more readable paraphrase of the above proposition is as follows:
\begin{prop}[Unique Readability (readable)]
  A term $M$ can only be one of the following forms:
  \begin{enumerate}
    \item[\rule{VAR}] $x$, where $x$ is a variable uniquely determined by $M$.
    \item[\rule{ABS}] $(\lambd[x][N])$, where $x$ is a variable and $N$ is
      another term, both of which is uniquely determined by $M$.
    \item[\rule{APP}] $(PQ)$, where $P$ and $Q$ are two terms uniquely
      determined by $M$.
  \end{enumerate}
\end{prop}

\end{document}